\documentclass{beamer}\usepackage[]{graphicx}\usepackage[]{color}
%% maxwidth is the original width if it is less than linewidth
%% otherwise use linewidth (to make sure the graphics do not exceed the margin)
\makeatletter
\def\maxwidth{ %
  \ifdim\Gin@nat@width>\linewidth
    \linewidth
  \else
    \Gin@nat@width
  \fi
}
\makeatother

\definecolor{fgcolor}{rgb}{0.345, 0.345, 0.345}
\newcommand{\hlnum}[1]{\textcolor[rgb]{0.686,0.059,0.569}{#1}}%
\newcommand{\hlstr}[1]{\textcolor[rgb]{0.192,0.494,0.8}{#1}}%
\newcommand{\hlcom}[1]{\textcolor[rgb]{0.678,0.584,0.686}{\textit{#1}}}%
\newcommand{\hlopt}[1]{\textcolor[rgb]{0,0,0}{#1}}%
\newcommand{\hlstd}[1]{\textcolor[rgb]{0.345,0.345,0.345}{#1}}%
\newcommand{\hlkwa}[1]{\textcolor[rgb]{0.161,0.373,0.58}{\textbf{#1}}}%
\newcommand{\hlkwb}[1]{\textcolor[rgb]{0.69,0.353,0.396}{#1}}%
\newcommand{\hlkwc}[1]{\textcolor[rgb]{0.333,0.667,0.333}{#1}}%
\newcommand{\hlkwd}[1]{\textcolor[rgb]{0.737,0.353,0.396}{\textbf{#1}}}%

\usepackage{framed}
\makeatletter
\newenvironment{kframe}{%
 \def\at@end@of@kframe{}%
 \ifinner\ifhmode%
  \def\at@end@of@kframe{\end{minipage}}%
  \begin{minipage}{\columnwidth}%
 \fi\fi%
 \def\FrameCommand##1{\hskip\@totalleftmargin \hskip-\fboxsep
 \colorbox{shadecolor}{##1}\hskip-\fboxsep
     % There is no \\@totalrightmargin, so:
     \hskip-\linewidth \hskip-\@totalleftmargin \hskip\columnwidth}%
 \MakeFramed {\advance\hsize-\width
   \@totalleftmargin\z@ \linewidth\hsize
   \@setminipage}}%
 {\par\unskip\endMakeFramed%
 \at@end@of@kframe}
\makeatother

\definecolor{shadecolor}{rgb}{.97, .97, .97}
\definecolor{messagecolor}{rgb}{0, 0, 0}
\definecolor{warningcolor}{rgb}{1, 0, 1}
\definecolor{errorcolor}{rgb}{1, 0, 0}
\newenvironment{knitrout}{}{} % an empty environment to be redefined in TeX

\usepackage{alltt} 
% \usepackage{graphicx}
\usepackage{graphics}
\usepackage[T1]{fontenc}
\usepackage{verbatim}
\usepackage{etoolbox}
\usepackage{hyperref}
\usepackage{color}
\makeatletter
\preto{\@verbatim}{\topsep=-6pt \partopsep=-6pt }
\makeatother
%\usepackage{fix-cm}
\setbeamercovered{transparent}


\renewcommand{\ni}{\noindent}


% \SweaveOpts{cache=TRUE, background="white"}


\title[3-Intro to ddply]{Introduction to ddply}
\subtitle{Cleaning and Summarizing Data}
\date{\hspace{1in}}
\institute[ISU]{Iowa State University}
\graphicspath{{figures/}}
\IfFileExists{upquote.sty}{\usepackage{upquote}}{}
\begin{document}

\begin{frame}
\maketitle
\end{frame}



\begin{frame}
\frametitle{Outline}
\begin{itemize}
\item conditionals \& subsets\medskip
\item \texttt{for} loops\medskip
\item avoiding \texttt{for} loops with \texttt{ddply}\medskip
\end{itemize}
\end{frame}


\begin{frame}[fragile]
\frametitle{Baseball Data}
\begin{itemize}
\item The \texttt{plyr} package contains the data set \texttt{baseball}
\item seasonal batting statistics of all major league players (through 2007)
\end{itemize}
\begin{knitrout}\scriptsize
\definecolor{shadecolor}{rgb}{1, 1, 1}\color{fgcolor}\begin{kframe}
\begin{alltt}
\hlkwd{library}\hlstd{(plyr)}
\hlkwd{help}\hlstd{(baseball)}
\hlkwd{head}\hlstd{(baseball)}
\end{alltt}
\end{kframe}
\end{knitrout}
\begin{knitrout}\tiny
\definecolor{shadecolor}{rgb}{1, 1, 1}\color{fgcolor}\begin{kframe}
\begin{verbatim}
           id year stint team lg  g  ab  r  h X2b X3b hr rbi sb cs bb so ibb hbp sh sf gidp
4   ansonca01 1871     1  RC1    25 120 29 39  11   3  0  16  6  2  2  1  NA  NA NA NA   NA
44  forceda01 1871     1  WS3    32 162 45 45   9   4  0  29  8  0  4  0  NA  NA NA NA   NA
68  mathebo01 1871     1  FW1    19  89 15 24   3   1  0  10  2  1  2  0  NA  NA NA NA   NA
99  startjo01 1871     1  NY2    33 161 35 58   5   1  1  34  4  2  3  0  NA  NA NA NA   NA
102 suttoez01 1871     1  CL1    29 128 35 45   3   7  3  23  3  1  1  0  NA  NA NA NA   NA
106 whitede01 1871     1  CL1    29 146 40 47   6   5  1  21  2  2  4  1  NA  NA NA NA   NA
\end{verbatim}
\end{kframe}
\end{knitrout}
\end{frame}

\begin{frame}[fragile]
\frametitle{Baseball Data}
\begin{itemize}
\item We would like to create career summary statistics for each player
\item Plan: subset on a player, and compute statistics
\end{itemize}
\begin{knitrout}\scriptsize
\definecolor{shadecolor}{rgb}{1, 1, 1}\color{fgcolor}\begin{kframe}
\begin{alltt}
\hlstd{ss} \hlkwb{<-} \hlkwd{subset}\hlstd{(baseball, id}\hlopt{==}\hlstr{"sosasa01"}\hlstd{)}
\hlkwd{head}\hlstd{(ss)}
\end{alltt}
\end{kframe}
\end{knitrout}
\vspace{-12pt}
\begin{knitrout}\tiny
\definecolor{shadecolor}{rgb}{1, 1, 1}\color{fgcolor}\begin{kframe}
\begin{verbatim}
            id year stint team lg   g  ab  r   h X2b X3b hr rbi sb cs bb  so ibb hbp sh sf gidp
66822 sosasa01 1989     1  TEX AL  25  84  8  20   3   0  1   3  0  2  0  20   0   0  4  0    3
66823 sosasa01 1989     2  CHA AL  33  99 19  27   5   0  3  10  7  3 11  27   2   2  1  2    3
67907 sosasa01 1990     1  CHA AL 153 532 72 124  26  10 15  70 32 16 33 150   4   6  2  6   10
69018 sosasa01 1991     1  CHA AL 116 316 39  64  10   1 10  33 13  6 14  98   2   2  5  1    5
70599 sosasa01 1992     1  CHN NL  67 262 41  68   7   2  8  25 15  7 19  63   1   4  4  2    4
71757 sosasa01 1993     1  CHN NL 159 598 92 156  25   5 33  93 36 11 38 135   6   4  0  1   14
\end{verbatim}
\end{kframe}
\end{knitrout}
\vspace{-12pt}
\begin{knitrout}\scriptsize
\definecolor{shadecolor}{rgb}{1, 1, 1}\color{fgcolor}\begin{kframe}
\begin{alltt}
\hlkwd{mean}\hlstd{(ss}\hlopt{$}\hlstd{h}\hlopt{/}\hlstd{ss}\hlopt{$}\hlstd{ab)}
\end{alltt}
\begin{verbatim}
## [1] 0.2682
\end{verbatim}
\end{kframe}
\end{knitrout}
\end{frame}

\begin{frame}[fragile]
\frametitle{Baseball Data}
\begin{itemize}
\item We would like to create career summary statistics for each player
\item Plan: subset on a player, and compute statistics
\end{itemize}

\begin{knitrout}\scriptsize
\definecolor{shadecolor}{rgb}{1, 1, 1}\color{fgcolor}\begin{kframe}
\begin{alltt}
\hlstd{ss} \hlkwb{<-} \hlkwd{subset}\hlstd{(baseball, id}\hlopt{==}\hlstr{"sosasa01"}\hlstd{)}
\hlkwd{head}\hlstd{(ss)}
\end{alltt}
\end{kframe}
\end{knitrout}
\vspace{-12pt}
\begin{knitrout}\tiny
\definecolor{shadecolor}{rgb}{1, 1, 1}\color{fgcolor}\begin{kframe}
\begin{verbatim}
            id year stint team lg   g  ab  r   h X2b X3b hr rbi sb cs bb  so ibb hbp sh sf gidp
66822 sosasa01 1989     1  TEX AL  25  84  8  20   3   0  1   3  0  2  0  20   0   0  4  0    3
66823 sosasa01 1989     2  CHA AL  33  99 19  27   5   0  3  10  7  3 11  27   2   2  1  2    3
67907 sosasa01 1990     1  CHA AL 153 532 72 124  26  10 15  70 32 16 33 150   4   6  2  6   10
69018 sosasa01 1991     1  CHA AL 116 316 39  64  10   1 10  33 13  6 14  98   2   2  5  1    5
70599 sosasa01 1992     1  CHN NL  67 262 41  68   7   2  8  25 15  7 19  63   1   4  4  2    4
71757 sosasa01 1993     1  CHN NL 159 598 92 156  25   5 33  93 36 11 38 135   6   4  0  1   14
\end{verbatim}
\end{kframe}
\end{knitrout}
\vspace{-12pt}
\begin{knitrout}\scriptsize
\definecolor{shadecolor}{rgb}{1, 1, 1}\color{fgcolor}\begin{kframe}
\begin{alltt}
\hlkwd{mean}\hlstd{(ss}\hlopt{$}\hlstd{h}\hlopt{/}\hlstd{ss}\hlopt{$}\hlstd{ab)}
\end{alltt}
\begin{verbatim}
## [1] 0.2682
\end{verbatim}
\end{kframe}
\end{knitrout}
{\large We need an automatic way to calculate this}
\end{frame}

\begin{frame}[fragile]
\frametitle{\texttt{for} loops}
\begin{itemize}
\item Idea: repeat the same (set of) statement(s) for each element of an index set
\item Setup: 
\begin{itemize}
\item Introduce counter variable (sometimes named \texttt{i})
\item Reserve space for results
\end{itemize}
\item Generic Code:
\end{itemize}

\begin{verbatim}
result <- rep(NA, length(indexset))
for(i in indexset){
  ... some statments ...
  result[i] <- ...
}
\end{verbatim}
\end{frame}

\begin{frame}[fragile]
\frametitle{\texttt{for} loops for Baseball}
\begin{itemize}
\item Index set: player id
\item Setup: 
\end{itemize}
\begin{knitrout}\scriptsize
\definecolor{shadecolor}{rgb}{1, 1, 1}\color{fgcolor}\begin{kframe}
\begin{alltt}
\hlcom{# Index set}
\hlstd{players} \hlkwb{<-} \hlkwd{unique}\hlstd{(baseball}\hlopt{$}\hlstd{id)}
\hlstd{n} \hlkwb{<-} \hlkwd{length}\hlstd{(players)}

\hlcom{# Place to store data}
\hlstd{ba} \hlkwb{<-} \hlkwd{rep}\hlstd{(}\hlnum{NA}\hlstd{, n)}

\hlcom{# Loop}
\hlkwa{for}\hlstd{(i} \hlkwa{in} \hlnum{1}\hlopt{:}\hlstd{n)\{}
  \hlstd{career} \hlkwb{<-} \hlkwd{subset}\hlstd{(baseball, id}\hlopt{==}\hlstd{players[i])}
  \hlstd{ba[i]} \hlkwb{<-} \hlkwd{with}\hlstd{(career,} \hlkwd{mean}\hlstd{(h}\hlopt{/}\hlstd{ab,} \hlkwc{na.rm}\hlstd{=T))}
\hlstd{\}}

\hlcom{# Results}
\hlkwd{summary}\hlstd{(ba)}
\end{alltt}
\begin{verbatim}
##    Min. 1st Qu.  Median    Mean 3rd Qu.    Max.    NA's 
##   0.000   0.183   0.246   0.223   0.270   0.500       6
\end{verbatim}
\end{kframe}
\end{knitrout}
\end{frame}

\begin{frame}[fragile]
\frametitle{\texttt{for} loops for Baseball}
\begin{itemize}
\item Index set: player id
\item i=0
\end{itemize}
\begin{knitrout}\scriptsize
\definecolor{shadecolor}{rgb}{1, 1, 1}\color{fgcolor}\begin{kframe}
\begin{alltt}
\hlcom{# Index set}
\hlstd{players} \hlkwb{<-} \hlkwd{unique}\hlstd{(baseball}\hlopt{$}\hlstd{id)}
\hlstd{n} \hlkwb{<-} \hlkwd{length}\hlstd{(players)}

\hlcom{# Place to store data}
\hlstd{ba} \hlkwb{<-} \hlkwd{rep}\hlstd{(}\hlnum{NA}\hlstd{, n)}

\hlkwd{head}\hlstd{(ba)}
\end{alltt}
\begin{verbatim}
## [1] NA NA NA NA NA NA
\end{verbatim}
\end{kframe}
\end{knitrout}
\end{frame}

\begin{frame}[fragile]
\frametitle{\texttt{for} loops for Baseball}
\begin{itemize}
\item Index set: player id
\item i=1
\end{itemize}
\begin{knitrout}\scriptsize
\definecolor{shadecolor}{rgb}{1, 1, 1}\color{fgcolor}\begin{kframe}
\begin{alltt}
\hlcom{# Index set}
\hlstd{players} \hlkwb{<-} \hlkwd{unique}\hlstd{(baseball}\hlopt{$}\hlstd{id)}
\hlstd{n} \hlkwb{<-} \hlkwd{length}\hlstd{(players)}

\hlcom{# Place to store data}
\hlstd{ba} \hlkwb{<-} \hlkwd{rep}\hlstd{(}\hlnum{NA}\hlstd{, n)}

\hlcom{# Loop}
\hlkwa{for}\hlstd{(i} \hlkwa{in} \hlnum{1}\hlopt{:}\hlnum{1}\hlstd{)\{}
  \hlstd{career} \hlkwb{<-} \hlkwd{subset}\hlstd{(baseball, id}\hlopt{==}\hlstd{players[i])}
  \hlstd{ba[i]} \hlkwb{<-} \hlkwd{with}\hlstd{(career,} \hlkwd{mean}\hlstd{(h}\hlopt{/}\hlstd{ab,} \hlkwc{na.rm}\hlstd{=T))}
\hlstd{\}}
\hlstd{i}
\end{alltt}
\begin{verbatim}
## [1] 1
\end{verbatim}
\begin{alltt}
\hlkwd{head}\hlstd{(ba)}
\end{alltt}
\begin{verbatim}
## [1] 0.3371     NA     NA     NA     NA     NA
\end{verbatim}
\end{kframe}
\end{knitrout}
\end{frame}

\begin{frame}[fragile]
\frametitle{\texttt{for} loops for Baseball}
\begin{itemize}
\item Index set: player id
\item i=2
\end{itemize}
\begin{knitrout}\scriptsize
\definecolor{shadecolor}{rgb}{1, 1, 1}\color{fgcolor}\begin{kframe}
\begin{alltt}
\hlcom{# Index set}
\hlstd{players} \hlkwb{<-} \hlkwd{unique}\hlstd{(baseball}\hlopt{$}\hlstd{id)}
\hlstd{n} \hlkwb{<-} \hlkwd{length}\hlstd{(players)}

\hlcom{# Place to store data}
\hlstd{ba} \hlkwb{<-} \hlkwd{rep}\hlstd{(}\hlnum{NA}\hlstd{, n)}

\hlcom{# Loop}
\hlkwa{for}\hlstd{(i} \hlkwa{in} \hlnum{1}\hlopt{:}\hlnum{2}\hlstd{)\{}
  \hlstd{career} \hlkwb{<-} \hlkwd{subset}\hlstd{(baseball, id}\hlopt{==}\hlstd{players[i])}
  \hlstd{ba[i]} \hlkwb{<-} \hlkwd{with}\hlstd{(career,} \hlkwd{mean}\hlstd{(h}\hlopt{/}\hlstd{ab,} \hlkwc{na.rm}\hlstd{=T))}
\hlstd{\}}
\hlstd{i}
\end{alltt}
\begin{verbatim}
## [1] 2
\end{verbatim}
\begin{alltt}
\hlkwd{head}\hlstd{(ba)}
\end{alltt}
\begin{verbatim}
## [1] 0.3371 0.2489     NA     NA     NA     NA
\end{verbatim}
\end{kframe}
\end{knitrout}
\end{frame}

\begin{frame}[fragile]
\frametitle{Your Turn}
\begin{itemize}
\item MLB rules for the greatest all-time hitters are that players have to have played at least 1000 games with at least as many at-bats in order to be considered\bigskip\bigskip
\item Extend the for loop above to collect the additional information\\ Introduce and collect data for two new variables: \texttt{games} and \texttt{atbats}
\end{itemize}
\end{frame}


\begin{frame}[fragile]
\frametitle{How did it go? What was difficult?}
\begin{itemize}
\item household chores (declaring variables, setting values each time) distract from real work\bigskip\bigskip
\item indices are error-prone\bigskip\bigskip
\item loops often result in slow code because R can compute quantities using entire vectors in an optimized way
\end{itemize}
\end{frame}

\begin{frame}[fragile]
\frametitle{plyr package}
\begin{itemize}
\item Routines from the plyr package help us to avoid for loops\bigskip
\item usage:
\begin{knitrout}\scriptsize
\definecolor{shadecolor}{rgb}{1, 1, 1}\color{fgcolor}\begin{kframe}
\begin{verbatim}
ddply(.data, .variables, .fun=NULL, ...)
\end{verbatim}
\end{kframe}
\end{knitrout}
\bigskip
\item Split-apply-combine approach:
\begin{enumerate}
\item SPLIT data into subsets on each element of an index set
\item APPLY the same statements for each suubset
\item COMBINE the results into a new data frame
\end{enumerate}
\end{itemize}
\end{frame}

\begin{frame}[fragile]
\frametitle{Example}
\begin{itemize}
\item Goal: Compute mean statistics for each player
\item Split the dataset by player ID, compute the mean for each column
\end{itemize}
\begin{knitrout}\scriptsize
\definecolor{shadecolor}{rgb}{1, 1, 1}\color{fgcolor}\begin{kframe}
\begin{alltt}
\hlstd{allstats} \hlkwb{<-} \hlkwd{ddply}\hlstd{(baseball,} \hlkwd{.}\hlstd{(id), mean)}

\hlkwd{head}\hlstd{(allstats)}
\end{alltt}
\begin{verbatim}
##          id V1
## 1 aaronha01 NA
## 2 abernte02 NA
## 3 adairje01 NA
## 4 adamsba01 NA
## 5 adamsbo03 NA
## 6 adcocjo01 NA
\end{verbatim}
\end{kframe}
\end{knitrout}
{\large What went wrong? \\\hfill Not all data is numeric!}
\end{frame}

\begin{frame}[fragile]
\frametitle{Summarize}
\begin{itemize}
\item A special function: \texttt{summarise} or \texttt{summarize}
\end{itemize}
\begin{knitrout}\scriptsize
\definecolor{shadecolor}{rgb}{1, 1, 1}\color{fgcolor}\begin{kframe}
\begin{alltt}
\hlkwd{summarise}\hlstd{(baseball,} \hlkwc{ab}\hlstd{=}\hlkwd{mean}\hlstd{(h}\hlopt{/}\hlstd{ab,} \hlkwc{na.rm}\hlstd{=T))}
\end{alltt}
\begin{verbatim}
##      ab
## 1 0.234
\end{verbatim}
\begin{alltt}
\hlkwd{summarize}\hlstd{(baseball,}
          \hlkwc{ba} \hlstd{=} \hlkwd{mean}\hlstd{(h}\hlopt{/}\hlstd{ab,} \hlkwc{na.rm}\hlstd{=T),}
          \hlkwc{games} \hlstd{=} \hlkwd{sum}\hlstd{(g,} \hlkwc{na.rm}\hlstd{=T),}
          \hlkwc{hr} \hlstd{=} \hlkwd{sum}\hlstd{(hr,} \hlkwc{na.rm}\hlstd{=T),}
          \hlkwc{ab} \hlstd{=} \hlkwd{sum}\hlstd{(ab,} \hlkwc{na.rm}\hlstd{=T))}
\end{alltt}
\begin{verbatim}
##      ba   games     hr      ab
## 1 0.234 1580070 113577 4891061
\end{verbatim}
\begin{alltt}
\hlkwd{summarize}\hlstd{(}\hlkwd{subset}\hlstd{(baseball, id}\hlopt{==}\hlstr{"sosasa01"}\hlstd{),}
          \hlkwc{ba} \hlstd{=} \hlkwd{mean}\hlstd{(h}\hlopt{/}\hlstd{ab,} \hlkwc{na.rm}\hlstd{=T),}
          \hlkwc{games} \hlstd{=} \hlkwd{sum}\hlstd{(g,} \hlkwc{na.rm}\hlstd{=T),}
          \hlkwc{hr} \hlstd{=} \hlkwd{sum}\hlstd{(hr,} \hlkwc{na.rm}\hlstd{=T),}
          \hlkwc{ab} \hlstd{=} \hlkwd{sum}\hlstd{(ab,} \hlkwc{na.rm}\hlstd{=T))}
\end{alltt}
\begin{verbatim}
##       ba games  hr   ab
## 1 0.2682  2354 609 8813
\end{verbatim}
\end{kframe}
\end{knitrout}
\end{frame}

\begin{frame}[fragile]
\frametitle{ddply + Summarize}
{\large A powerful combination to create summary statistics}
\begin{knitrout}\scriptsize
\definecolor{shadecolor}{rgb}{1, 1, 1}\color{fgcolor}\begin{kframe}
\begin{alltt}
\hlstd{careers} \hlkwb{<-} \hlkwd{ddply}\hlstd{(baseball,} \hlkwd{.}\hlstd{(id), summarize,}
                 \hlkwc{ba} \hlstd{=} \hlkwd{mean}\hlstd{(h}\hlopt{/}\hlstd{ab,} \hlkwc{na.rm}\hlstd{=T),}
                 \hlkwc{games} \hlstd{=} \hlkwd{sum}\hlstd{(g,} \hlkwc{na.rm}\hlstd{=T),}
                 \hlkwc{homeruns} \hlstd{=} \hlkwd{sum}\hlstd{(hr,} \hlkwc{na.rm}\hlstd{=T),}
                 \hlkwc{atbats} \hlstd{=} \hlkwd{sum}\hlstd{(ab,} \hlkwc{na.rm}\hlstd{=T))}

\hlkwd{head}\hlstd{(careers)}
\end{alltt}
\begin{verbatim}
##          id     ba games homeruns atbats
## 1 aaronha01 0.3011  3298      755  12364
## 2 abernte02 0.1824   681        0    181
## 3 adairje01 0.2363  1165       57   4019
## 4 adamsba01 0.2097   482        3   1019
## 5 adamsbo03 0.2378  1281       37   4019
## 6 adcocjo01 0.2752  1959      336   6606
\end{verbatim}
\end{kframe}
\end{knitrout}
\end{frame}

\begin{frame}[fragile]
\frametitle{Your Turn}
\begin{itemize}
\item Find some summary statistics for each of the teams (variable \texttt{team})\medskip
\begin{itemize}
\item How many different (\texttt{unique}) players has the team had?
\item What was the team's first/last season?
\end{itemize}\medskip\bigskip
\item Challenge: \\
Find the number of players on each team over time. Does the number change?
\end{itemize}

\begin{knitrout}\scriptsize
\definecolor{shadecolor}{rgb}{1, 1, 1}\color{fgcolor}\begin{kframe}
\begin{alltt}
\hlstd{careers} \hlkwb{<-} \hlkwd{ddply}\hlstd{(baseball,} \hlkwd{.}\hlstd{(id), summarize,}
                 \hlkwc{ba} \hlstd{=} \hlkwd{mean}\hlstd{(h}\hlopt{/}\hlstd{ab,} \hlkwc{na.rm}\hlstd{=T),}
                 \hlkwc{games} \hlstd{=} \hlkwd{sum}\hlstd{(g,} \hlkwc{na.rm}\hlstd{=T),}
                 \hlkwc{homeruns} \hlstd{=} \hlkwd{sum}\hlstd{(hr,} \hlkwc{na.rm}\hlstd{=T),}
                 \hlkwc{atbats} \hlstd{=} \hlkwd{sum}\hlstd{(ab,} \hlkwc{na.rm}\hlstd{=T))}

\hlkwd{head}\hlstd{(careers)}
\end{alltt}
\end{kframe}
\end{knitrout}
\end{frame}
\end{document}
